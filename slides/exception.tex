\section{Mehr zu exceptions}


\subsection{Exception Guarantees}


\begin{frame}{Exception Guarantees}

    Sind nötig bei: 
	\begin{itemize}
		\item Funktionen die Objekte verändern
		\item Memberfunktionen von Klassen
		\item Konstruktoren
		\item Destruktoren
	\end{itemize}
	
	Mögliche Stufen:
	\begin{enumerate}
		\item \enquote{exception unsafe}: ungültige pointer, memleaks, unlogische zustände von objekten
		\item \enquote{basic/weak guarantee}: keine memleaks, objekt kann noch benutzt werden, zustand unbekannt
		\item \enquote{strong guarantee}: wenn eine exception geworfen wurde, wurde nichts verändert
		\item \enquote{no-throw guarantee}: es wird keine exception geworfen
	\end{enumerate}

\end{frame}


\begin{frame}{Destruktor}

    Der Destruktor darf keine Exception werfen (no-throw guarantee).
    
    Wenn ein Destruktor eine Exception wirft, wird das Programm abgebrochen.
    
    Im Destruktor dürfen Funktionen die Exceptions werfen aufgerufen werden, solange die Exceptions abgefanegen werden.

\end{frame}

\begin{frame}{Konstruktor}

    Der Konstruktor sollte dann eine Exception werfen, wenn das Objekt nicht in einen gültigen Zustand versetzt werden kann.
    
    Allokierter Speicher muss vorher freigegeben werden.
    
    Veränderte andere Objekte müssen vorher in den Ausgangszustand zurück versetzt werden.
    
    strong guarantee
    \lstinputlisting{cpp-code/except1.cpp}

\end{frame}


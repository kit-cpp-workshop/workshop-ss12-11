\section{Dokumentation}

\begin{frame}{Warum Dokumentation wichtig ist}
	Beispiel: \texttt{std::map::insert} \\
	\texttt{pair<iterator,bool> insert(const pair<const Key, T> \&x)}
	
	\begin{itemize}
		\item Vorbedingungen?
		\item Mögliche Fehlerzustände?
		\item Rückgabewerte?
		\item Seiteneffekte?
		\item Sonstiges Verhalten?
		\item Laufzeit?
	\end{itemize}
\end{frame}

\begin{frame}{Doxygen}
	\begin{itemize}
		\item Ein Werkzeug zur Dokumentation
		\item Dokumentation erfolgt mit speziellen Kommentaren im Quellcode
		\begin{itemize}
			\item Vorteil: Lokalität, Wartbarkeit
			\item Verschiedene Stil wählbar (Javadoc, Qt)
		\end{itemize}
		\item \texttt{doxygen} durchsucht Quelltext nach Dokumentation
		\item Steuerung über ein \enquote{doxyfile} (Suchpfade, Formatierung etc.)
		\item Ausgabeformate: PDF, HTML, RTF, Manpage \dots
	\end{itemize}

	\footnotesize{\url{http://www.stack.nl/~dimitri/doxygen/}}
\end{frame}

\begin{frame}{Beispiel}
	\lstinputlisting{cpp-code/doxygen.cpp}
	
	(Javadoc-Style, \texttt{JAVADOC\_AUTOBRIEF = YES})
\end{frame}


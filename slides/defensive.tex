\section{Defensive Programming}
\begin{frame}{Paranoia!}
	\begin{itemize}
		\begin{small}
			\item Murphys Gesetz: \enquote{Whatever can go wrong, will go wrong.}
			\item \enquote{Programming today is a race between software engineers striving to build bigger and better idiot-proof programs and the universe, trying to produce bigger and better idiots. So far, the universe wins.} [Rich Cook]
		\end{small}
	\end{itemize}
\end{frame}

\begin{frame}{Defensive Programmierung?!}	
	\begin{block}{Ziele}
		Verbesserung von Software im Bezug auf:
		\begin{itemize}
			\item Sicherheit
			\item Stabilität
			\item Robustheit
		\end{itemize}
	\end{block}
	
	\begin{itemize}
		\item Guter Code ist gut lesbar (und damit überprüfbar)
		\item Vermeiden von Fehlerquellen
		\item (Frühzeitige) Erkennung/Behandlung möglicher Fehlerfälle
		\begin{itemize}
			\item Dazu gehört auch: Fehler nicht vertuschen!
		\end{itemize}
		\item Schutz von Invarianten
		\item \enquote{never trust the client}
	\end{itemize}
\end{frame}

\begin{frame}{Design by Contract}
	\begin{itemize}
		\item \enquote{Vertrag} zwischen Schnittstellen/Klassen/Methoden und Nutzern über deren Verhalten und Verwendung
		\item Schlüsselelemente: Invarianten
		\begin{itemize}
			\item Vor- und Nachbedingungen
			\item Invarianten von Klassen/Algorithmen
		\end{itemize}
		\item Weitere Aussagen/Zusicherungen:
		\begin{itemize}
			\item Zulässige Eingaben, mögliche Ausgaben
			\item Sonstige Seiteneffekte
			\item Mögliche Fehlerzustände/Exceptions
			\item Leistungsgarantien
		\end{itemize}
	\end{itemize}
\end{frame}

\begin{frame}{Design by Contract bedeutet \dots}
	\begin{block}{\dots für den Klassenentwickler}
		\begin{itemize}
			\item Seine Vorraussetzungen/Bedingungen zu dokumentieren
			\begin{itemize}
				\item Und im Sinne defensiver Programmierung auch durchzusetzen!
			\end{itemize}
			\item Eigenen Code anhand ausgearbeiteter Invarianten zu verifizieren
			\item Implementierungsdetails zu verbergen (information hiding)
			\begin{itemize}
				\item Alles von außen Sichtbare gehört zum Vertrag
				\item Implementierungen können sich ändern - Die Schnittstelle nicht
			\end{itemize}
		\end{itemize}
	\end{block}
	\begin{block}{\dots für den Nutzer}
		\begin{itemize}
			\item Einhaltung der dokumentierten Bedingungen
			\item Ausschließliche (!) Nutzung auf Basis des Vertrages
			\begin{itemize}
				\item Nach außen verborgene Implementierungsdetails gehören nicht dazu!
			\end{itemize}
		\end{itemize}
	\end{block}
\end{frame}

\begin{frame}[fragile]{Assertions}
	\begin{itemize}
		\item Aussagen die bei einem korrekten Programm immer wahr sein sollen
		\item Fehlschlagen einer Assertion $\rightarrow$ Sofortiges Programmende!
		\item Dokumentieren und überprüfen Invarianten
	\end{itemize}
	
	\begin{block}{Verwendung}
		\begin{itemize}
			\item Einbindung: \verb|#include <cassert>|
			\begin{itemize}
				\item Standardmäßig sind Assertions aktiv
				\item Deaktivierung: \verb|#define NDEBUG| vor dem \verb|include|
				\begin{itemize}
					\item[] Besser/Bequemer: Über Compiler-Flag (\verb|-DNDEBUG|)
				\end{itemize}
			\end{itemize}
			\item Einsatz: \verb|assert(23 == 42);|
		\end{itemize}
	\end{block}
	
	\begin{block}{Assertions vs. Exceptions}
		\begin{itemize}
			\item Exceptions behandeln Fehlerfälle und Vertragsverletzungen
			\item Assertions überwachen Korrektheit von Annahmen/Vorraussetzungen
		\end{itemize}
	\end{block}
\end{frame}

